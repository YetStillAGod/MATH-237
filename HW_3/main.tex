\documentclass{article}
\usepackage{amsmath, amssymb , amsthm, graphicx,mathtools}

\begin{document}

\section*{Question 1}

\begin{enumerate}
    \item If $g$ generates $G$ of order $n$, then $g^k$ is a generator $\iff \gcd(n,k)=1$:
    \begin{proof}
    ~
        \begin{enumerate}
            \item ord$(g^k)=\frac{n}{\gcd(n,k)}$:
            \begin{enumerate}
                \item $t\coloneqq \frac{n}{\gcd(n,k)},d\coloneqq \gcd(n,k)$, then $n=dt    ,k=k_0d,\gcd(t,k_0)=1$;
                \item $(g^k)^t=g^{kt}=g^{k_0dt}=g^{k_0n}=(g^n)^{k_0}=e$;
                \item  So ord$(g^k)\mid t$, ord$(g^k)\leq t$;
                \item Suppose $g^{kt_0}=e$, then $n\mid kt_0$;
                \item $\gcd(n,k)=d$, so $t\mid k_0t_0$;
                \item Since $\gcd(t,k_0)=1$, $t\mid t_0$;
                \item $t\leq t_0$;
                \item The minimal value of $t_0$ is ord$(g^k)$, so $t\leq\text{ord}(g^k)$;
                \item So $\text{ord}(g^k)=t=\frac{n}{\gcd(n,k)}$;
            \end{enumerate}
            \item $g^k$ is a generator $\iff ord(g^k)=n \iff \gcd(n,k)=1$.
        \end{enumerate}
    \end{proof}
    \item $\mathbb{F}^*_5$:
    \begin{enumerate}
        \item $|\mathbb{F}^*_5|=4$, so the generator must also have order 4;
        \item $2^1=2,2^2=4,2^3=3,2^4=1$, so 2 is a generator of $\mathbb{F}^*_5$;
        \item $\gcd(k,4)=1\implies k\in\{1,3\}$;
        \item So $2^1=2,2^3=3$ are the generators;
    \end{enumerate}
    \item $\mathbb{F}^*_{19}$:
    \begin{enumerate}
        \item ord$(a)=n\iff \forall p|n$ ($p$ is a prime)$,a^{n/p}\ne1$;
        \begin{proof}
        ~
            \begin{enumerate}
                \item $\Rightarrow$:
                \begin{enumerate}
                    \item Suppose ord$(a)=n,p|n$ is a prime;
                    \item $a^n=a^{n/q\cdot q}=(a^{n/q})^q=1$;
                    \item So ord$(a^{n/q})=q$ since $n$ is the smallest number $k$ satisfies $a^k=1$;
                \end{enumerate}
                \item $\Leftarrow$:
                \begin{enumerate}
                    \item $m\coloneqq $ord$(a)$, so $m\mid n,m\leq n$;
                    \item If $m<n$, $\exists q:q\mid n/m\implies kq=n/m$ for some $k$;
                    \item $km=n/q$, so $n/q$ is a multiple of $m$;
                    \item $a^{n/q}=1\nLeftrightarrow a^{n/q}\ne1$;
                    \item So $m=n=\text{ord}(a)$.
                \end{enumerate}
                \item So ord$(a)=n\iff \forall p|n$ ($p$ is a prime)$,a^{n/p}\ne1$
            \end{enumerate}
        \end{proof}
        \item $|\mathbb{F}_{19}^*|=18=2\cdot 3^2$, so just check if $a^9\ne 1$ and $a^6\ne 1$;
        \item $2^9=18\ne1, 2^6=7\ne1$;
        \item So 2 is a generator;
        \item So $2^k$ is a generator for all $\gcd(k,18)=1,k\in \{1,5,7,11,13,17\}$;
        \item So generators are $\{2^1,2^5,2^7,2^{11},2^{13},2^{17}\}=\{2,13,14,15,3,10\}$.
    \end{enumerate}
\end{enumerate}

\newpage

\section*{Question 2}

\begin{enumerate}
    \item 
    \begin{enumerate}
        \item \begin{align*}
        &x^4+3x^3+5x^2+9x+6\\
        =&x^2(x^2+3)+3x^3+2x^2+9x+6\\
        =&x^2(x^2+3)+3x(x^2+3)+2x^2+6\\
        =&x^2(x^2+3)+3x(x^2+3)+2(x^2+3)\\
        =&(x^2+3x+2)(x^2+3)\\
    \end{align*}
    \item $q(x)=x^2+3x+2,r(x)=0$
    \end{enumerate}
    \item 
    \begin{enumerate}
        \item $3x^2+2x-3\equiv 3x^2+3x+4$
        \item \begin{align*}
            &x^6+3x^5+4x^2-3x+2\\
            \equiv&x^6+3x^5+4x^2+4x+2\\
            \equiv&5x^4(3x^2+2x+4)+x^4+4x^2+4x+2,(3^{-1}=5,-20\equiv 1)\\
            \equiv&5x^4(3x^2+2x+4)+5x^2(3x+2x+4)+4x^3+5x^2+4x+2,(-10\equiv 4)\\
            \equiv&5x^4(3x^2+2x+4)+5x^2(3x+2x+4)+6x(3x^2+2x+4)+x+2,(18\equiv 4,12\equiv 5,24\equiv 3)\\
            =&(5x^4+5x^2+6x)(3x^2+2x+4)+x+2\\
        \end{align*}
        \item $q(x)=5x^4+4x^2+6x,r(x)=x+2$
    \end{enumerate}
\end{enumerate}

\newpage

\section*{Question 3}

\begin{enumerate}
    \item In a polynomial field $F$: $\deg fg=\deg f+\deg g,f,g\ne 0$:
    \begin{proof}
    ~
        \begin{enumerate}
            \item Write $f(x)=\sum_{i=0}^m a_ix^i,g(x)=\sum_{i=0}^n b_ix^i,h(x)=f(x)g(x)$;
            \item $\deg fg\geq \deg f+\deg g$:
            \begin{enumerate}
                \item At $m+n$ term in $h(x)$, the coefficient is $a_mb_n$;
                \item Since $a_m,b_n\ne0\in F$, a field is an integral domain, $a_mb_n\ne 0$;
                \item So $x^{m+n}$ term appears with coefficient $a_mb_n$, and $\deg fg\geq \deg f+\deg g$;
            \end{enumerate}
            \item $\deg fg\leq \deg f+\deg g$:
            \begin{enumerate}
                \item $\nexists i,j,i\leq m,j\leq n:i+j>m+n$;
                \item So the greatest degree possible for $h(x)$ is $m+n$;
                \item $\deg fg\leq \deg f+\deg g$
            \end{enumerate}
            \item So $\deg fg=\deg f+\deg g$.
        \end{enumerate}
    \end{proof}
    \item $\mathbb{Z}_4[x]$:
    \begin{enumerate}
        \item Let $f(x)$ be a unit in $\mathbb{Z}_4[x]$;
        \item construct $\pi:\mathbb{Z}_4[x]\to\mathbb{F}_2[x]$ by reducing every coefficient mod 2;
        \item $\pi$ is a ring homomorphism since the addition between different terms are preserved as well as the multiplication between $x^i$ and the coefficients, and modding is a homomorphism;
        \item $f(x)$ is a unit, so $\exists g(x)\in\mathbb{Z}_4[x]:f(x)g(x)=1$;
        \item $\pi(f(x)g(x))=\pi(f(x))\pi(g(x))=\pi(1)=1$;
        \item So $\pi(f(x))$ is a unit;
        \item Unit in $\mathbb{F}_2[x]$ is only 1 since $\deg 1=\deg f+\deg g=0$, so $\deg f=\deg g=0$, and they are both constants;
        \item So all the units of $\mathbb{Z}_4[x]$ are 1 modding 2;
        \item So the units of $\mathbb{Z}_4[x]$ are the polynomials that the constants are 1 or 3 and other coefficients are 2 or 0;
        \item The units are $\{a_0+2g(x),a_0\in\{1,3\},g(x)\in\mathbb{Z}_4[x]\}$;
    \end{enumerate}
    \item $\mathbb{F}_4[x]$:
    \begin{enumerate}
        \item Suppose $f(x)$ is aa unit, then $\exists g(x)\in\mathbb{F}_5[x]:f(x)g(x)=1$;
        \item In $\mathbb{F}_5$, the non-zero elements are all units;
        \item So the units in $\mathbb{F}_5[x]$ are non-zero constants$\{1,2,3,4\}$ since $\deg 1=\deg f+\deg g=0$, so $\deg f=\deg g=0$, and they are both constants.
    \end{enumerate}
\end{enumerate}

\newpage

\section*{Question 4}

\begin{enumerate}
    \item $x^2+4$:
    \begin{align*}
        &x^4+4\\
        \equiv&x^4-1\\
        =&(x^2+1)(x^2-1)\\
        \equiv&(x^2-4)(x^2-1)\\
        =&(x+2)(x-2)(x+1)(x-1)\\
        \equiv&(x+2)(x+3)(x+1)(x+4)\\
    \end{align*}
    \item $x^4+3x^2+2x+4$:
    \begin{enumerate}
        \item The possible roots are $0,1,2,3,4$, test the roots:
        \begin{enumerate}
            \item $x=0:x^4+3x^2+2x+4=4$;
            \item $x=1:x^4+3x^2+2x+4=10\equiv 0$;
            \item $x=2:x^4+3x^2+2x+4=36\equiv 1$;
            \item $x=3:x^4+3x^2+2x+4=118\equiv 3$;
            \item $x=4:x^4+3x^2+2x+4=316\equiv 1$;
        \end{enumerate}
        \item So $x=1$ is a root, divide by $x-1$:
        \begin{align*}
            &x^4+3x^2+2x+4\\
            =&(x-1)x^3+x^3+3x^2+2x+4\\
            =&(x-1)x^3+(x-1)x^2+4x^2+2x+4\\
            =&(x-1)x^3+(x-1)x^2+(x-1)4x+6x+4\\
            \equiv&(x-1)x^3+(x-1)x^2+(x-1)4x+x-1\\
            =&(x-1)(x^3+x^2+4x+1)\\
            \equiv&(x+4)(x^3+x^2+4x+1)\\
        \end{align*}
        \item Test the roots of $x^3+x^2+4x+1$:
        \begin{enumerate}
            \item $x=0:x^3+x^2+4x+1=1$
            \item $x=1:x^3+x^2+4x+1=7\equiv2$
            \item $x=2:x^3+x^2+4x+1=21\equiv 1$
            \item $x=3:x^3+x^2+4x+1=49\equiv 4$
            \item $x=4:x^3+x^2+4x+1=97\equiv 2$
        \end{enumerate}
        \item So $x^3+x^2+4x+1$ is irreducible in $\mathbb{F}_5$;
        \item $x^4+3x^2+2x+4=(x+4)(x^3+x^2+4x+1)$
    \end{enumerate}
     \item $x^3+2x+3$:
     \begin{enumerate}
         \item Test the roots:
         \begin{enumerate}
             \item $x=0:x^3+2x+3=3$
             \item $x=1:x^3+2x+3=6\equiv 1$
             \item $x=2:x^3+2x+3=15\equiv 0$
         \end{enumerate}
         \item So $x=2$ is a root, divide by $x-2$:
         \begin{align*}
             &x^3+2x+3\\
             =&(x-2)x^2+2x^2+2x+3\\
             =&(x-2)x^2+(x-2)2x+6x+3\\
             \equiv&(x-2)x^2+(x-2)2x+x-2\\
             =&(x-2)(x^2+2x+1)\\
             =&(x-2)(x+1)^2\\
             \equiv&(x+3)(x+1)^2\\
         \end{align*}
     \end{enumerate}
     \item $x^3+3x+2$:
    \begin{enumerate}
        \item Test the roots of $x^3+3x+2$:
        \begin{enumerate}
         \item $x=0:x^3+3x+2=2$
         \item $x=1:x^3+3x+2=6\equiv 1$
         \item $x=2:x^3+3x+2=16\equiv 1$
         \item $x=3:x^3+3x+2=38\equiv 3$
         \item $x=4: x^3+3x+2=78\equiv 3$
        \end{enumerate}
        \item So there are no roots of $x^3+3x+2$ in $\mathbb{F}_5$, hence irreducible.
    \end{enumerate}
\end{enumerate}

\newpage

\section*{Question 5}

\begin{proof}
~
    \begin{enumerate}
        \item $\phi:\mathbb{Z}\to\mathbb{Z}_m$ by reduction $\mod{m}$;
        \item $\pi:\mathbb{Z}[x]\to\mathbb{Z}_m[x],\phi(\sum a^ix^i)=\sum\phi(a_i)x^i$;
        \item Surjective ring homomorphism:
        \begin{enumerate}
            \item During this mapping, the addition between different terms are preserved as well as the multiplication between $x^i$ and the coefficients, and $\phi$ is a homomorphism, so it is a ring homomorphism;
            \item $\forall g(x)=\sum b_ix^i\in\mathbb{Z}_m,\exists f(x)=\sum a_ix^i\in\mathbb{Z}:a_i\equiv b_i\mod{m}$;
            \item So $\pi$ is a surjective ring homomorphism;
        \end{enumerate}
        \item if $\deg \pi(f)=\deg f$, then if $\pi(f)$ is irreducible, $f$ is irreducible:
        \begin{enumerate}
            \item Suppose $\deg \pi(f)=\deg f$ and $\pi(f)$ is irreducible;
            \item Assume $f$ is reducible;
            \item $\exists g(x),h(x)\in\mathbb{Z}[x]:f=gh,\deg g,\deg h\geq1$;
            \item $\pi(f)=\pi(g)\pi(h)$;
            \item WLOG, If $\pi(g)$ has coefficient of highest degree $\equiv 0\mod{m}$, $\deg\pi(f)<\deg f\nLeftrightarrow\deg\pi(f)=\deg f$;
            \item So $\deg \pi(g)\geq1,\deg\pi(h)\geq 1$, and $\pi(f)$ factors nontrivially, causing contradiction;
            \item $f$ is irreducible.
        \end{enumerate}
    \end{enumerate}
\end{proof}

\newpage

\section*{Question 6}

\begin{proof}
    ~\begin{enumerate}
    \item $x^3+10x^2+15x+5$:
    \begin{enumerate}
        \item Use Eisenstein with $p=5$:
        \item $5\mid 5,10,15, 25\nmid 5,5\nmid 1$;
        \item So $x^3+10x^2+15x+5$ is irreducible;
    \end{enumerate}
    \item $x^3+x^2+x+2$:
    \begin{enumerate}
        \item Use Root test: $x=\pm 1,\pm2$:
        \begin{enumerate}
            \item $x=1:x^3+x^2+x+2=5$
            \item $x=-1: x^3+x^2+x+2=1$
            \item $x=2: x^3+x^2+x+2=16$
            \item $x=-2,x^3+x^2+x+2=-4$
        \end{enumerate}
        \item There are no root in $\mathbb{Z}$, so it is not divisible over $\mathbb{Q}$;
    \end{enumerate}
    \item $x^3+17x+36$:
    \begin{enumerate}
        \item Use Root test: $x=\pm1,\pm2,\pm3,\pm4,\pm6,\pm9,\pm12,\pm18,\pm36$:\
        \begin{enumerate}
            \item $x=1:x^3+17x+36=54$
            \item $x=-1:x^3+17x+36=18$
            \item $x=2:x^3+17x+36=78$
            \item $x=-2:x^3+17x+36=-6$
            \item $x=3:x^3+17x+36=114$
            \item $x=-3:x^3+17x+36=-42$
            \item $x=4:x^3+17x+36=168$
            \item $x=-4:x^3+17x+36=-96$
            \item $x=6:x^3+17x+36=356$
            \item $x=-6:x^3+17x+36=-282$
            \item $x=9:x^3+17x+36=918$
            \item $x=-9:x^3+17x+36=-846$
            \item $x=12:x^3+17x+36=1968$
            \item $x=-2:x^3+17x+36=-1896$
            \item $x=18:x^3+17x+36=6174$
            \item $x=-18:x^3+17x+36=-6102$
            \item $x=36:x^3+17x+36=47304$
            \item $x=-36:x^3+17x+36=-47232$
        \end{enumerate}
        \item There are no root in $\mathbb{Z}$, so it is not divisible over $\mathbb{Q}$.
    \end{enumerate}
\end{enumerate}
\end{proof}
\end{document}