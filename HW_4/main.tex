
\documentclass[11pt]{article}
\usepackage[margin=1in]{geometry}
\usepackage{amsmath,amsthm,amssymb,mathtools,enumitem}
\usepackage{microtype}
\usepackage{hyperref}
\hypersetup{colorlinks=true,linkcolor=black,urlcolor=blue}

\newcommand{\ZZ}{\mathbb{Z}}
\newcommand{\QQ}{\mathbb{Q}}
\newcommand{\FF}{\mathbb{F}}
\newcommand{\ord}{\text{ord}}

\begin{document}

\section*{Question 1}

\begin{enumerate}
    \item No.
    \item Counterexample: $V_4=\ZZ/2\ZZ\times \ZZ/2\ZZ$:
    \begin{enumerate}
        \item Proper subgroups: $\langle(1,0)\rangle,\langle(0,1)\rangle,\langle(1,1)\rangle$;
        \item All the proper subgroups are cyclic with order 2;
        \item But $V_4$ dos not have an element of order 4, hence it is not cyclic.
    \end{enumerate}
\end{enumerate}

\newpage

\section*{Question 2}

\begin{proof}
~
    \begin{enumerate}
        \item $|G|=\Pi_{i=1}^rp_i^{a_i}$, and by assumption, there is a unique subgroup $P_t$ of order ${p_t}^{a_t}$;
        \item Suffice to prove when $|G|=p^n$:
        \begin{enumerate}
            \item $\forall g\in G, gP_tg^{-1}\leq G, \ord (gP_tg^{-1})=\ord(P_t)$, and by uniqueness $gP_tg^{-1}=P_t$, so $P_t\trianglelefteq G$;
            \item $x\in P_i,y\in P_j, x^{-1}y^{-1}xy\in P_i\cap P_j$ since they are normal, but $P_i\cap P_j=\{1\}(\gcd({p_i}^{a_i},{p_j}^{a_j})=1)$;
            \item So $x^{-1}y^{-1}xy=1$ and it is commutative;
            \item So $G\cong \Pi_{i=1}^r P_i$, and it suffices for the theorem just to prove the case $|G|=p^n$.
        \end{enumerate}
        \item Induction on $n$:
        \begin{enumerate}
            \item Base Case: $|G|=p,p^2,p^3$:
            \begin{enumerate}
                \item $|G|=p$: All of such are cyclic;
                \item $|G|=p^2:$ There are two isomorphisms: $C_{p^2},C_p\times C_p$, but for $C_p\times C_p$, the subgroups of order $p$ is not unique (in fact, there are more than $p$ subgroups), so $G\cong C_{p^2}$ ad hence cyclic;
                \item $|G|=p^3:$ Set $H\leq G,|H|=p$, then $|G/H|=p^2$ by Lagrange Theorem, by the previous two, $H$ and $G/H$ are cyclic, then consequently, $p^3$ is cyclic.
            \end{enumerate}
            \item Induction Hypothesis: $|H|=p^{n-1}$ is cyclic$\implies |G|=p^{n}$ is cyclic;
            \item Induction Step: $|G|=p^{n+1}$:
            \begin{enumerate}
                \item Set $H\leq G, |H|=p^n$, by Lagrange Theorem, $|G/H|=p$;
                \item By base case, $G/H$ is cyclic, and by Induction Hypothesis, $H$ is cyclic, by the subgroup and its quotient group both cyclic
                \item Then $G$ is cyclic (similar to $|G|=p^3$ case in base case);
            \end{enumerate}
            \item So $G$ is cyclic for any $n:|G|=p^n$;
            \item By the sufficient condition, $G$ has a subgroup for every order $d\mid|G|$,$G$ is cyclic.
        \end{enumerate}
    \end{enumerate}
\end{proof}

\newpage

\section*{Question 3}

\begin{enumerate}
    \item $\langle f\rangle$ is prime $\iff$ $f$ is irreducible:
    \begin{proof}
        ~
        \begin{enumerate}
            \item $\Rightarrow$:
            \begin{enumerate}
                \item Suppose $f=ab$ with $\deg a,\deg b\geq 1$;
                \item $\deg a,\deg b<\deg f$;
                \item $\langle f\rangle$ since generated by $f$, $\nexists g\in \langle f\rangle:\deg g<\deg f$;
                \item So $a,b\notin \langle f\rangle$, but $f\in\langle f\rangle$;
                \item $\langle f\rangle$ is not prime;
                \item Taking contrapositive: $\langle f\rangle$ is prime $\implies$ $f$ is irreducible;
            \end{enumerate}
            \item $\Leftarrow$:
            \begin{enumerate}
                \item $f$ is irreducible;
                \item So $\langle f\rangle$ is a maximal ideal;
                \item Maximal ideal in a commutative ring with unity is also prime, so $\langle f\rangle$ is prime;
            \end{enumerate}
            \item So $\langle f\rangle$ is prime $\iff$ $f$ is irreducible.
        \end{enumerate}
    \end{proof}
    \item $p$ is irreducible and $p\mid rs\implies p\mid r\lor p\mid s$:
    \begin{proof}
        ~
        \begin{enumerate}
            \item $p$ is irreducible, so $\langle p\rangle$ is prime by (a);
            \item $p\mid rs$, so $rs\in\langle p\rangle$;
            \item $\langle p\rangle$ is prime, so $rs\in\langle p\rangle\implies r\in\langle p\rangle\lor s\in\langle p\rangle$;
            \item So $r\in\langle p\rangle\lor s\in\langle p\rangle$;
            \item This is equivalent to $p\mid r\lor p\mid s$.
        \end{enumerate}
    \end{proof}
    \item Unique Factorization:
    \begin{proof}
        ~
        \begin{enumerate}
            \item Existence:
            \begin{enumerate}
                \item If $f$ is reducible, then by division algorithm, $\exists g,h:f=gh,\deg g,\deg h<\deg f$;
                \item This process can continue until irreducibles;
            \end{enumerate}
            \item Uniqueness:
            \begin{enumerate}
                \item Suppose $f=\Pi_{i=1}^ru_ip_i=\Pi_{j=1}^sv_jq_j$ where $u,v$ are constants and $q,j$ are irreducibles;
                \item According to b, $p_1\mid q_j$ for some $j$, then since $p_1$ and $q_j$ are both irreducible, $\exists u,v:up_1=vq_j$;
                \item Then $p_1$ and $q_j$ cancel out, and same for other $p_i$;
                \item so uniqueness conserves.
            \end{enumerate}
            \item So every non-constant polynomial factors uniquely as a product of irreducible polynomials.
        \end{enumerate}
    \end{proof}
    \item $(x-1)^3(x+1)=(x-1)^2(2x-2)(3x+3)$:
    \begin{enumerate}
        \item \begin{align*}
        &(x-1)^2(2x-2)(3x+3)\\
        =&(x-1)^22(x-1)(3x+3)\\
        =&(x-1)^3(6x+6)\\
        \equiv&(x-1)^3(x+1)\\
    \end{align*}
    \item So $(x-1)^3(x+1)$ and $(x-1)^2(2x-2)(3x+3)$ are actually the same factorization.
    \end{enumerate}
\end{enumerate}

\newpage

\section*{Question 4}

\begin{enumerate}
    \item $a=\sqrt{1+3\sqrt{2}}$:
    \begin{enumerate}
        \item $3\sqrt{2}=a^2-1$, so $(a^2-1)^2=18$ and $a^4-2a^2-17=0$
        \item $f(x)\coloneqq x^4-2x^2-17\in\QQ[x]$;
        \item By rational root test $\{\pm1,\pm17\}$, $f(x)$ has no root over $\QQ$, hence no linear factor;
        \item By quadratics factorization: 
        \begin{enumerate}
            \item $x^4-2x^2-17=(x^2+ax+b)(x^2+cx+d)=x^4+(a+c)x^3+(ac+b+d)x^2+(ad+bc)+bd$;
        \item So $a+c=0,ac+b+d=-2,ad+bc=0,bd=-17$;
        \item By substitution $a=-c$, $a(b-d)=0$:
        \begin{enumerate}
            \item $a=0\implies ac+b+d=b+d=-2$; 
            \item $b=d\implies bd=b^2=-17$;
            \item Both impossible by  $(b,d)\in\{(1,-17),(-1,17),(17,-1),(-17,1)\} $;
        \end{enumerate}
        \end{enumerate}
        \item So $f$ is irreducible in $\QQ[x]$;
        \item So $irr(a,\QQ)=x^4-2x^2-17,\deg (a/\QQ)=4$.
    \end{enumerate}
    \item $a=\sqrt{2}+i$
    \begin{enumerate}
        \item \begin{align*}
            a-\sqrt{2}=&i\\
            (a-\sqrt{2})^2+1=&0\\
            ((a+\sqrt{2})^2+1)((a-\sqrt{2})^2+1)=&0\\
            ((a+\sqrt{2})(a-\sqrt{2}))^2+(a+\sqrt{2})^2+(a-\sqrt{2})^2+1=&0\\
            (a^2-2)^2+2a^2+4+1=&0\\
            a^4-4a^2+4+2a^2+4+1=&0\\
            a^4-2a^2+9=&0\\
        \end{align*}
        \item $f(x)\coloneqq x^4-2x^2+9\in\QQ[x]$;
        \item By rational root test $\{\pm 1,\pm3\pm9\}$, $f(x)$ has no root over $\QQ$, hence no linear factor;
        \item by quadratics factorization:
        \begin{enumerate}
            \item $x^4-2x^2+9=(x^2+ax+b)(x^2+cx+d)=x^4+(a+c)x^3+(ac+b+d)x^2+(ad+bc)+bd$;
            \item So $a+c=0,ac+b+d=-2,ad+bc=0,bd=9$;
            \item Substituting $a=-c, a(b-d)=0$:
            \begin{enumerate}
                \item $a=0\implies ac+b+d=b+d=-2$;
                \item $b=d\implies -a^2+2b=-2\land b^2=9$;
                \item Both impossible by $(b,d)\in\{(1,9),(9,1),(3,3),(-3,-3)\}$
            \end{enumerate}
        \end{enumerate}
        \item So $f$ is irreducible in $\QQ[x]$;
        \item So $irr(a,\QQ)=x^4-2x^2+9,\deg (a/\QQ)=4$.
    \end{enumerate}
\end{enumerate}

\newpage

\section*{Question 5}

\begin{enumerate}
    \item $(a^2+a+1)(a^2+a)$:
    \begin{enumerate}\
        \item $a^3+a^2+a+2=0\implies a^3=-a^2-a-2$
        \item \begin{align*}
            (a^2+a+1)(a^2+a)=&a^4+2a^3+2a^2+a\\
            =&a(-a^2-a-2)+2(-a^2-a-2)+2a^2+a\\
            =&-a^3-a^2-3a-2\\
            =&-(-a^2-a-2)-a^2-3a-4\\
            =&-2a-2\\
        \end{align*}
        \item $(a^2+a+1)(a^2+a)=-2a-2$\\
    \end{enumerate}
    \item $(a-1)^{-1}$:
    \begin{enumerate}
        \item Find $Ax^2+Bx+C:(a-11)(Ax^2+Bx+C)=1$
        \item $f(x)\coloneqq x^3+x^2+x+2$;
        \item \begin{align*}
            x^3+x^2+x+2=&x^2(x-1)+2x^2+x+2\\
            =&x^2(x-1)+2x(x-1)+3x+2\\
            =&x^2(x-1)+2x(x-1)+3(x-1)+5\\
            =&(x-1)(x^2+2x+3)+5\\
        \end{align*}
        \item $f(x)=(x-1)(x^2+2x+3)+5$
        \item $f(a)=(a-1)(a^2+2a+3)+5=0$
        \item $(a-1)(-\frac{1}{5}(a^2+2a+3))=1$
        \item So $(a-1)^{-1}=-\frac{1}{5}(a^2+2a+3)$
    \end{enumerate}
\end{enumerate}

\end{document}
